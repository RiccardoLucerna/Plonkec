\documentclass[a4paper, 11pt]{article}
\usepackage[utf8]{inputenc}
\usepackage[T1]{fontenc}
\usepackage[slovene]{babel}
\usepackage{lmodern}

\usepackage{amsmath}
\usepackage{amsthm}
\usepackage{amsfonts}

% definicija novega ukaza
\newcommand{\N}{\mathbb{N}}

{\theoremstyle{definition}
\newtheorem{definicija}{Definicija}}

\title{Peanovi aksiomi}
\author{Mojca Novak}
\date{27. avgust 2024}

\begin{document}

    \maketitle

    \begin{abstract}
        V nadaljevanju je zapisana definicija Peanovih aksiomov.
    \end{abstract}

    \begin{definicija}
        \textbf{Peanovi aksiomi \cite{zapiski}}
        Množica naravnih števil je množica $\N$ s funkcijo $\varphi : \N \rightarrow \N$, 
        ki vsakemu naravnemu številu $n$ priredi njegovega neposrednega naslednika $\varphi(n)$. 
        Pri tem veljajo naslednji aksiomi:
        \begin{enumerate}
            \item $\N$ vsebuje število $\epsilon$, ki ni neposredni naslednik nobenega naravnega števila;
            \item neposredna naslednika dveh različnih naravnih števil sta različna, tj.\ funkcija $\varphi$ je injektivna: 
            če je $n \neq m$, je $\varphi(n) \neq \varphi(m)$;
            \item Če za podmnožico $A \in \N$ veljata lastnosti:
            \begin{enumerate}
                \item $\epsilon \in A$ in
                \item če je $n \in A$, je tudi $\varphi(n) \in A$,
                potem je $A = \N$.
            \end{enumerate}
        \end{enumerate}
        % konec definicije
    \end{definicija}

    \bibliographystyle{plain}
    \bibliography{viri.bib}
    
\end{document}